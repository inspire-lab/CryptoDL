\subsection*{Backend}

This project uses \href{https://github.com/nlohmann/json}{\tt nlohmann\textquotesingle{}s json parser} to load json files and create models from them using our architecture.

\subsection*{Custom \hyperlink{classActivation}{Activation} Functions}

To use custom activation functions, extend the \hyperlink{classActivation}{Activation} class in your Keras model through Python. e.\+g.\+: 
\begin{DoxyCode}
# extend activation class
class Square(Activation):
    def \_\_init\_\_(self, activation, **kwargs):
        super(Square, self).\_\_init\_\_(activation, **kwargs)
        self.\_\_name\_\_ = 'square'

# activation function
def square(x):
    return K.square(x)

# update custom objects
keras.utils.get\_custom\_objects().update( \{ 'square': Square(square) \} )
\end{DoxyCode}
 Then, you will need to add the appropriate name in \hyperlink{JSONModel_8h_source}{J\+S\+O\+N\+Model.\+h}\textquotesingle{}s {\ttfamily grab\+Activation()} and function in {\ttfamily \hyperlink{ActivationFunctionImpl_8h_source}{architecture/\+Activation\+Function\+Impl.\+h}}.

\subsection*{Memory Usage}

When creating a model using \hyperlink{JSONModel_8h_source}{J\+S\+O\+N\+Model.\+h}\textquotesingle{}s {\ttfamily from\+File()}, you can dictate what type of memory usage you would like. Although we currently only allow for greedy usage (and it acts as a default param for creation), there are plans to allow for a lazy memory usage instantiation.

\subsection*{\hyperlink{classLayer}{Layer} Attributes}

As we currently only allow for Conv2D, \hyperlink{classDense}{Dense}, and \hyperlink{classFlatten}{Flatten} layer functionality, these are the attributes that we will focus on.

\subsubsection*{Conv2D}

The Keras json file provides us with the following attributes\+: {\ttfamily kernel\+\_\+initializer, bias\+\_\+regularizer, dilation\+\_\+rate, activity\+\_\+regularizer, activation, data\+\_\+format, filters, kernel\+\_\+regularizer, use\+\_\+bias, kernel\+\_\+constraint, name, bias\+\_\+initializer, kernel\+\_\+size, dtype, trainable, batch\+\_\+input\+\_\+shape, bias\+\_\+constraint, strides, padding}

We only use the following attributes in our architecture\+: {\ttfamily name, activation, filters, kernel\+\_\+size, strides, padding} and sometimes {\ttfamily batch\+\_\+input\+\_\+shape} (when it is the first layer in the model)

\subsubsection*{\hyperlink{classDense}{Dense}}

The Keras json file provides us with the following attributes\+: {\ttfamily bias\+\_\+regularizer, units, activation, bias\+\_\+initializer, dtype, kernel\+\_\+regularizer, use\+\_\+bias, kernel\+\_\+constraint, kernel\+\_\+initializer, name, trainable, bias\+\_\+constraint, activity\+\_\+regularizer, batch\+\_\+input\+\_\+shape}

We only use the following attributes in our architecture\+: {\ttfamily name, activation, units} and sometimes {\ttfamily batch\+\_\+input\+\_\+shape} (when it is the first layer in the model)

\subsubsection*{\hyperlink{classFlatten}{Flatten}}

The Keras json file provides us with the following attributes\+: {\ttfamily data\+\_\+format, trainable, name}

We only use the following attributes in our architecture\+: {\ttfamily name} 